% Created 2020-08-10 Mon 15:24
% Intended LaTeX compiler: pdflatex
\documentclass[11pt]{article}
\usepackage[utf8]{inputenc}
\usepackage[T1]{fontenc}
\usepackage{graphicx}
\usepackage{grffile}
\usepackage{longtable}
\usepackage{wrapfig}
\usepackage{rotating}
\usepackage[normalem]{ulem}
\usepackage{amsmath}
\usepackage{textcomp}
\usepackage{amssymb}
\usepackage{capt-of}
\usepackage{hyperref}
\author{Koen Colpaert}
\date{\today}
\title{Security Officer meeting reports \& notes}
\hypersetup{
 pdfauthor={Koen Colpaert},
 pdftitle={Security Officer meeting reports \& notes},
 pdfkeywords={},
 pdfsubject={},
 pdfcreator={Emacs 26.1 (Org mode 9.4)}, 
 pdflang={English}}
\begin{document}

\maketitle
\hypersetup{colorlinks=true,linkcolor=darkblue}
\newpage

\section{\textit{<2020-07-14 Tue> } MadCap}
\label{sec:orgb3ad8e6}

\subsection{Algemeen}
\label{sec:orgf148986}

\begin{center}
\begin{tabular}{lll}
Naam & Entiteit & Functie\\
\hline
Koen Colpaert & POEMA & CISO\\
Tia Cormon & POEMA & DPO\\
Lenn Depoorter & VLABEL & Medewerker Kenniscel\\
\end{tabular}
\end{center}

\subsection{Probleemstelling}
\label{sec:orgab12d2c}
Voor het publiceren van handboeken/kennisdeling maakt de Kenniscel van Vlabel gebruik van een toepassing 
``MadCap Flare''. ``Madcap Flare'' staat in voor het genereren van de webpagina's vanuit XML-pagina's die in eerste
instantie op lokale storage opgeslagen staan. Deze MadCap-projecten worden zowel als bronbestand en als eindproduct
opgeslagen. 

In een eerste iteratie werden de bestanden opgeslagen op de fileserver maar sinds de uitfasering van de M-schijven 
gebeurt dit via de SharePoint. Daar waar het hele proces van opslag > generatie > opslag/publicatie vroeger een tiental
minuten in beslag nam is dit nu opgelopen tot een half uur. Om die reden is de Kenniscel geïnteresseerd in de mogelijkheden
van de cloud oplossing van dezelfde leverancier; ``MadCap Central''.

Een eerste overleg vond plaats tussen de Kenniscel en de CISO/DPO van POEMA om de risico's inzake cloudmigratie te bespreken. 
Daarbij gaf de Kenniscel enige toelichting over de huidige procedure en de inhoud van de website. Het gebruik is eerder beperkt
te noemen met slechts een 5-tal medewerkers en 2 beheerders. 
Privacy-gegevens worden niet verwerkt maar wel informatie die betrekking heeft op de algemene werking van Vlabel. Voor zover wij
konden nagaan bevat de website een schat aan informatie waarvan sommige als vertrouwelijk kan beschouwd worden.

Voor een aantal technische vragen moest de Kenniscel het antwoord schuldig blijven. De CISO zal contact opnemen met de leverancier
voor meer info over:
\begin{itemize}
\item het opslaan van de XML project-bestanden: staan deze servers in Europa? Encryptie? Backup? Vernietiging bestanden na einde overeenkomst?
\item gebruikersbeheer: hoe worden gebruikers aangemaakt? Username/paswoord of ook MFA?
\end{itemize}

Eventueel kan als alternatief ook overwogen worden om de oude workflow via de M-schijf terug in te voeren aangezien vooral de 
SharePoint de vertragende factor blijkt te zijn. Daarnaast is uit testen ook gebleken dat de generatie van de webdocumentatie via
de cloudoplossing ook een deel sneller gaat. Dat laatste kan misschien opgevangen worden door de MadCap Central op een VMware-machine
te laten draaien met aparte storage voor de XML-projecten.

\subsection{Risicobeheersing}
\label{sec:org0077ee7}
Niettegenstaande de gepubliceerde handboeken en documentatie geen persoonsgebonden gegevens bevatten is een beperkte DPIA wel
aangewezen aangezien de toepassing wel (technische) informatie bevat over de door Vlabel gebruikte toepassingen alsook motivaties
voor beslissingen bij bepaalde procedures.
Het SO beveelt ook de nodige beheersmaatregelen aan inzake het opvolgen van de toegangen en gebruikers.

\subsection{Beslissing CISO/DPO}
\label{sec:orgefe1c9d}
tbd
\end{document}
